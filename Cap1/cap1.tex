Esta seção começa com a motivação para o trabalho. Em seguida, tem-se o objetivo do projeto e a organização deste documento.

\section{Motivação}

A gestão do material carga da Força Aérea é feita com o apoio do Sistema
Integrado de Logística de Material e Serviços - SILOMS, mantido pelo Centro de
Computação da Aeronáutica do Rio de Janeiro - CCA RJ. Hoje este sistema atende de
forma eficaz às necessidades das Organizações Militares (OM) quanto à gestão, entretanto
existem algumas restrições:

\begin{itemize}

	\item A conferência do material permanente é realizada de forma
manual por militares da OM. Depois de localizada a etiqueta, a
descrição é conferida com a relação de material setorial disponibilizada pela
seção de registro. Este processo é demorado, cansativo - dependendo da
quantidade de itens - e sujeito a inúmeros erros durante a execução.

	\item A movimentação dos materiais é dada por um processo
administrativo moroso, no qual passar por diversos agentes da OM, gerando
papel e, consequentemente, arquivos.

\end{itemize}

O SILOMS permite a impressão de etiquetas, com ou sem código de barras, e
permite, inclusão, depreciação, transferência para outros setores e/ou outras Unidades e
baixa. No entanto, as conferências são feitas manualmente por não haver equipamento que se
aproveite da existência dos códigos de barra.

\section{Objetivo}
A proposta de trabalho é a criação de um sistema complementar para gerir os materiais, utilizando-se dos dados do SILOMS e de um aplicativo de celular capaz de ler os códigos de barra já existentes, atualmente não utilizados pelo pessoal interno à OM. Ao ler o código de barras com o aplicativo, o usuário insere informações de localização e estado do material carga, e tais dados são enviados a um servidor local do qual se poderia extrair o relatório de maneira mais simples, poupando tempo e mão de obra . Inicialmente aplicado ao ICEA, o sistema poderia futuramente ser expandido para a FAB como um todo, inclusive integrado ao próprio SILOMS.

\section{Organização do trabalho}

\subsection{Decisões de Projeto}
O Capítulo \ref{chproj} descreve as decisões de projeto, mostrando as razões e os requisitos que levaram a tais escolhas.

\subsection{Desenvolvimento}
O Capítulo \ref{chdev} mostra as ferramentas utilizadas e alguns detalhes do desenvolvimento do projeto.

\subsection{Próximos Passos}
O Capítulo \ref{chroadmap} contém as ideias e projeções para a finalização do projeto, além de correções e refatorações no que já foi feito até o presente momento.