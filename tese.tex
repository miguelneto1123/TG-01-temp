%%% Exemplo de utilização da classe ITA
%%%
%%%   por        Fábio Fagundes Silveira   -  ffs [at] ita [dot] br
%%%              Benedito C. O. Maciel     -  bcmaciel [at] ita [dot] br
%%%              Giovani Volnei Meinertz   -  giovani [at] ita [dot] br
%%%    	         Hudson Alberto Bode       -  bode [at] ita [dot]br
%%%    	         P. I. Braga de Queiroz    -  pi [at] ita [dot] br
%%%    	         Jorge A. B. Gripp         -  gripp [at] ita [dot] br
%%%    	         Juliano Monte-Mor         -  jamontemor [at] yahoo [dot] com [dot] br
%%%    	         Tarcisio A. B. Gripp      -  tarcisio.gripp [at] gmail [dot] com
%%%    	         
%%%
%%%  IMPORTANTE: O texto contido neste exemplo nao significa absolutamente nada.  :-)
%%%              O intuito aqui eh demonstrar os comandos criados na classe e suas
%%%              respectivas utilizacoes.
%%%
%%%  Tese.tex  2016-08-25
%%%  $HeadURL: http://www.apgita.org.br/apgita/teses-e-latex.php $
%%%
%%% ITALUS
%%% Instituto Tecnológico de Aeronáutica --- ITA, Sao Jose dos Campos, Brasil
%%%                   http://groups.yahoo.com/group/italus/
%%% Discussion list: italus {at} yahoogroups.com
%%%
%++++++++++++++++++++++++++++++++++++++++++++++++++++++++++++++++++++++++++++++
% Para alterar o TIPO DE DOCUMENTO, preencher a linha abaixo \documentclass[?]{?}
%   \documentclass[tg]{ita}			= Trabalho de Graduacao
%   \documentclass[tgfem]{ita}	= Para Engenheiras
%   								msc     		= Dissertacao de Mestrado
%   								mscfem   		= Para Mestras
%   								dsc      		= Tese de Doutorado
%   								dscfem   		= Para Doutoras
%   								quali    		= Exame de Qualificacao
%   								qualifem 		= Exame de Qualificacao para Doutoras
% Para 'Draft Version'/'Versao Preliminar' com data no rodape, adicionar 'dv':
%   \documentclass[dsc, dv]{ita} 
% Para trabalhos em Inglês, adicionar 'eng':
%   \documentclass[dsc, eng]{ita}
%		\documentclass[dsc, eng, dv]{ita}
%++++++++++++++++++++++++++++++++++++++++++++++++++++++++++++++++++++++++++++++
\documentclass[tg]{ita}    % ITA.cls based on standard book.cls 
% Quando alterar a classe, por exemplo de [msc] para [msc, eng]) rode mais uma vez o botão BUILD OUTPUT caso haja erro
\usepackage{ae}
\usepackage{graphicx}
\usepackage{epsfig}
\usepackage{amsmath}
\usepackage{amssymb} 
\usepackage{subfig}
\usepackage{multirow}
\usepackage{float}
\usepackage{minted}

%++++++++++++++++++++++++++++++++++++++++++++++++++++++++++++++++++++++++++++++
% Espaçamento padrão de todo o documento
%++++++++++++++++++++++++++++++++++++++++++++++++++++++++++++++++++++++++++++++
\onehalfspacing

%singlespacing Para um espaçamento simples
%onehalfspacing Para um espaçamento de 1,5
%doublespacing Para um espaçamento duplo

%++++++++++++++++++++++++++++++++++++++++++++++++++++++++++++++++++++++++++++++
% Identificacoes (se o trabalho for em inglês, insira os dados em inglês)
% Para entradas abreviadas de Professora (Profa.) em português escreva: Prof$^\textnormal{a}$.
%++++++++++++++++++++++++++++++++++++++++++++++++++++++++++++++++++++++++++++++
\course{Engenharia de Computação} % Programa de PG ou Curso de Graduação

% Autor do trabalho: Nome Sobrenome
\authorgender{masc}                     %sexo: masc ou fem
\author{Miguel Macedo de Araújo}{Neto}
\itaauthoraddress{Rua H8A, 113}{12.228-460}{São José dos Campos--SP}

% Titulo da Tese/Dissertação
\title{Automatização do Conferimento Patrimonial do ICEA usando Smartphones Android}

% Orientador
\advisorgender{masc}                    % masc ou fem
\advisor{Prof.~Dr.}{José Maria Parente de Oliveira}{ITA}

% Coorientador (Caso não haja coorientador, colocar ambas as variáveis \coadvisorgender e \coadvisor comentadas, com um % na frente)
%\coadvisorgender{fem}									% masc ou fem
%\coadvisor{Prof$^\textnormal{a}$.~Dr$^\textnormal{a}$.}{Doralice Serra}{OVNI}

% Pró-reitor da Pós-graduação
%\bossgender{masc}												% masc ou fem
%boss{Prof.~Dr.}{John von Neumann}

%Coordenador do curso no caso de TG
\bosscoursegender{fem}									% masc ou fem
\bosscourse{Prof.~Dr.}{Cecília de Azevedo Castro César}

% Palavras-Chaves informadas pela Biblioteca -> utilizada na CIP
\kwcip{MAVLink}
%\kwcip{}
%\kwcip{}

% membros da banca examinadora

%\examiner{Prof. Dr.}{Alan Turing}{Presidente}{ITA}
%\examiner{Prof. Dr.}{Linus Torwald}{}{UXXX}
%\examiner{Prof. Dr.}{Richard Stallman}{}{UYYY}
%\examiner{Prof. Dr.}{Donald Duck}{}{DYSNEY}
%\examiner{Prof. Dr.}{Mickey Mouse}{}{DISNEY}

% Data da defesa (mês em maiúsculo, se trabalho em inglês, e minúsculo se trabalho em português) 
\date{11}{junho}{2018}

% Número CDU - (somente para TG)
\cdu{621.38}

% Glossario
\makeglossary
\frontmatter

\begin{document}
% Folha de Rosto e Capa para o caso do TG
\maketitle

% Dedicatoria: Nao esqueca essa secao  ... :-)
\begin{itadedication}
A todos aqueles que acreditaram e acreditam em mim
\end{itadedication}

% Agradecimentos
\begin{itathanks}
Primeiramente à minha família, que sempre esteve presente quando eu precisei, desde quando estava estudando para passar no vestibular.

Aos meus colegas de apartamento, moradores e agregados, que tornaram esses 5 anos uma experiência mais agradável e com certeza mais divertida.

Ao ICEA, na pessoa do Maj Av Anderson Raunaimer, que surgiu com a proposta de TG, e ao Prof Parente por ter acolhido a ideia.
\end{itathanks}

% Epígrafe
\thispagestyle{empty}
\ifhyperref\pdfbookmark[0]{\nameepigraphe}{epigrafe}\fi
\begin{flushright}
\begin{spacing}{1}
\mbox{}\vfill
{\sffamily\itshape
``As our smartphone becomes even smarter,\\
mobile technology should actually take\\
the burden out of our lives.''\\}
--- \textsc{Peggy Johnson}
\end{spacing}
\end{flushright}

% Resumo
%\begin{abstract}
%\noindent
%%\input{Cap0/resumo}
%%\end{abstract}

% Abstract
\begin{abstract}
\noindent
Micro Air Vehicle Link is an simple open source protocol that has become popular in applications envolving small unmanned aircraft as drones. The protocol is also known for the lack of strong security measures once that is focused on lightweight communication over a bandwith-constrained channel. This paper overviews the main security issues of the protocol presented in the literature as lack of encryption, implementation flaws and vulnerability to network attacks.
The analysis is supported by software-in-the-loop simulations using the Ardupilot project. Besides this work presents the main aspects to understand the MAVLink protocol and the steps necessary to configure the simulation environment in order to reproduce the analysis. 
\end{abstract}

% Lista de figuras
\listoffigures %opcional

% Lista de tabelas
%\listoftables %opcional

% Lista de abreviaturas
\listofabbreviations
\begin{longtable}{ll}
MAVLink & Micro Air Vehicle Link \\
RPA & Remotely Piloted Aircraft \\
RPAS & Remotely Piloted Aerial System \\

\end{longtable}

 %opcional

% Lista de simbolos
% \listofsymbols
% \input{Cap0/listasimbolos} %opcional

% Sumario
\tableofcontents

\mainmatter
% Os capitulos comecam aqui

\chapter{Introdução}
This section starts with the motivation for the work. It will follow by the objective of the research and by the work organization that presents the approach adopted to study the MAVLink Protocol.

\section{Motivation}
Remotely Piloted Aircraft Systems (RPAS) like drones have become more popular nowadays because they can provide a new way to solve human issues. Uses for drones vary in different practical applications as security services, emergency services, urban plannig, recreation, entertainment and communications.

As a more specific example of application of drones, ICEA (Instituto de Controle do Espaço Aéreo) researches about remote flight inspection using unmanned aircraft. Flight inspection operations are proccesses made to verify and to validate the availability, quality, accuracy and integrity of navigation aids, as PAPI (Precision Approach Path Indicator) and VOR (Very High Frequency Omnidirectional Range).

If you start to develop a project using RPAS you will probably need to search and to learn about Micro Air Vehicle Link (MAVLink), once that the communication protocol is a important part in applications with drones.
The MAVLink protocol was first released in 2009 by Lorenz Meier. At first the protocol was created for educational purposes, but today it is so popular that has become a world standard in projects envolving RPAs/drones.

Once that drones have become polular in commercial and security applications, they are systems susceptible to attacks. Besides some applications require security requirements that cannot match with MAVLink standards of security (for example MAVLink does not use encryption in its messages). So analyze security aspects of MAVLink is necessary to find and to understand vulnerabilities in the communication protocol.

%\begin{figure}[ht]
%\centering
%\includegraphics[width=0.5\textwidth]{Cap1/cupim}
%\caption{Proibido estacionar cupins. Legenda grande, com o objetivo de demonstrar a indentação na lista de figuras.}
%\label{cupim}
%\end{figure}

%vê o problema de controle tolerante a falhas através de uma perspectiva integrada, foi proposta por
%{marcel4}. Os autores apresentam um ambiente híbrido consistindo de três unidades básicas que garantem a compleição de tarefas na presença de qualquer número de juntas falhas (Figura \ref{cupim}). A primeira unidade é um esquema de detecção
%e isolação de falhas que continuamente monitora o manipulador para detectar e identificar possíveis falhas nas juntas. A segunda unidade é responsável pela reconfiguração do controle. A terceira unidade é composta de algorítmos de
%controle apropriados para cada tipo de configuração do robô, baseado na informação da unidade de reconfiguração \cite{COFFEE2000}.

%Segundo, o critério de otimização utilizado será o acoplamento entre as juntas do
%manipulador e neste caso, temos um sistema redundante quando ocorre falha de uma das juntas do manipulador de três juntas, e seu posicionamento é controlado pelas duas restantes. Nossa solução para o problema é baseada na formulação
%inversa ({nakamura}). A

%\begin{figure}[ht!]
%\centering
%\includegraphics[width=1\textwidth]{Cap1/cupimconcreto}
%\caption{Exemplo real de cupim frente ao seu dilema.}
%\label{FDII}
%\end{figure}


\section{Objective}
	One of the goals of this work is to reune reliable information about the MAVLink protocol in other to become a reference material. In gereral papers related to RPAS applications have a small section about this protocol, but more information is necessary  to undersatand how it works. Although the protocol has its own documentation, many aspects are not explained in details and it lacks a practical use example of the MAVLink inside a simulation environment. 
    
    The second goal is to provide a security analysis of MAVLink protocol using the main strategies commonly used in the literature. Besides the security conclusions, this work aims to show all the details and configurations of the tools used in the study.
%\cite{Nascimento1970}.

%\section{Related Work}

\section{Work Organization}
\subsection{Protocol Overview}
Chapter 2 reports the main aspects of MAVLink in order to contextualize new developers. This chapter aims to follow the MAVLink Developer Guide in order to clarify and complement the official documentation.

\subsection{Simulation Environment}
Chapter \ref{sec_setting_sitl} describe the steps to configure the Software-In-The-Loop enviroment, where the security simulations will take place.

\subsection{Security Analysis of MAVLink}
	This Chapter will start listing the main security issues of the protocol. It also shows different approaches to identify vulnerabilities and how to set up these approaches in a computer.
    
\subsection{New sections}
	New chapters will be included during the work progress. 

\chapter{Decisões de Projeto} \label{chproj}
\section{Requisitos gerais}

Para o desenvolvimento de um sistema complexo como o proposto, faz-se necessário levantar requisitos de projeto mínimos, visando a utilidade, a eficiência e a acessibilidade disponibilizadas para o usuário. Dados a motivação e o objetivo do trabalho, temos os seguintes requisitos gerais:

\begin{enumerate}

	\item Ter uma maneira acessível e eficiente de ler códigos de barra: decidiu-se pelo uso de smartphones Android, a variação de smartphone mais utilizada. Além disso, se houver necessidade de que a OM possua celulares próprios para o controle de material que não o dos militares responsáveis por essa tarefa, tais smartphones têm menor custo, facilitando uma possível compra;

	\item Centralizar a informação de todos os materiais: um dos desafios atuais no controle dos materiais é a compilação das informações. Com a quantidade de papel envolvida, proveniente de anotações de várias pessoas diferentes vistoriando seções da OM, a compilação fica susceptível a erros humanos, seja por interpretação ou falta de atenção. O sistema deve ser capaz de fazer a compilação por si só; e
    
    \item Facilitar a geração de relatórios: novamente, a compilação das informações exige juntar as entradas de uma planilha, escrevendo uma a uma. O sistema deve ser capaz de gerar tal informação com as informações já centralizadas.
    
\end{enumerate}

\section{Especificações adicionais}

Complementando os requisitos levantados, há ainda as seguintes informações sobre os códigos de barra:

\begin{itemize}

	\item Cada material possui um código de barras único, mesmo que seja idêntico a outro material (duas cadeiras de escritório do mesmo modelo possuem códigos diferentes, por exemplo), e o código de barras é traduzido exatamente para o número identificador do material, conforme a Figura \ref{projfig01} abaixo.
    
    \begin{figure}[ht!]
        \centering
        \includegraphics[width=0.75\textwidth]{Cap2/etiqueta}
        \caption{Modelo de etiqueta de material.}
        \label{projfig01}
    \end{figure}
    
    Pode-se ver que há também a informação da OM e da seção na etiqueta (respectivamente UG e Depend), além de uma breve descrição do material e de seu número identificador único.
    
    \item Cada OM possui um código identificador, assim como cada seção de cada OM.
    
\end{itemize}

\section{Decisões de programação}

\subsection{Programação do aplicativo}
Como o sistema operacional (SO) escolhido para os celulares foi o Android, havia a escolha entre Java e Kotlin, ambas linguagens aceitas pelo próprio Google (mantenedor do SO) como linguagens oficiais para programação em Android. Escolheu-se Kotlin pelas seguintes razões:

\begin{enumerate}

	\item Sintaxe: Kotlin foi construída com base no que os usuários de Java queriam de uma linguagem de programação. O resultado é uma linguagem mais atraente e simples de entender;
    
	\item Concisão: a linguagem reduz drasticamente a quantidade de \textit{boilerplate code}, isto é, seções de código que devem ser incluídas em muitos lugares com pouca ou nenhuma alteração, o que é bastante recorrente em Java;
    
    \item Interoperabilidade: a linguagem é mais recente que Java, e portanto a quantidade de bibliotecas produzidas para ela é bem menor. Porém, ela executa usando o mesmo ``motor'' que Java, o JVM, o que torna as duas linguagens completamente interoperáveis entre si;
    
    \item Segurança: a linguagem foi estruturada para evitar ao máximo a ocorrência \textit{exceptions} (erros durante a execução dos programas), principalmente um dos erros mais comuns de Java, o de ponteiro nulo.
    
\end{enumerate}

Kotlin é uma linguagem bem recente, a versão atual é 1.2.50. Para aprender a programar nessa linguagem, foram usados a documentação oficial da própria linguagem, o curso \cite{kotlinbeginners} para o básico da sintaxe e o curso \cite{kotlinandroid} para a programação própria do Android.

A versão mínima escolhida do Android que aceita o aplicativo é a versão 4.0.3 (Ice Cream Sandwich). Dessa forma, praticamente todos os dispositivos em circulação no mercado atualmente são abrangidos.

\subsection{Programação do banco de dados}

Integrar um sistema novo ou criar uma funcionalidade nova para o SILOMS exigiria muito mais tempo, principalmente devido a questões burocráticas. Com isso, foi decidido criar um sistema complementar ao SILOMS, que utilize dados dele e seja um servidor para todos os usuários do aplicativo desenvolvido. Escolheu-se criar uma API REST utilizando Node.js como \textit{back-end} do servidor (através das bibliotecas Loopback e Swagger) e MongoDB como fonte de dados.

Montar uma API REST foi a melhor opção encontrada para um servidor remoto de dados que alimenta um aplicativo Android, pois facilita a integração do banco de dados com o aplicativo e ainda permite que uma possível futura integração ao SILOMS ocorra facilmente, bastando apenas escrever as funções REST para o SILOMS servir como fonte de dados.

A escolha do Loopback é baseada na simplicidade: a biblioteca cria facilmente as rotas e comandos necessários para o funcionamento completo de uma estrutura CRUD (\textit{\textbf{C}reate, \textbf{R}ead, \textbf{U}pdate and \textbf{D}elete}), além de facilitar a integração com a fonte de dados e fornecer um sistema de autenticação. O Swagger cria uma interface gráfica para o servidor, permitindo o \textit{debug} visual das funções.

Simplicidade também foi o motivo da escolha do MongoDB. Os dados são armazenados em coleções, usando uma notação bastante semelhante ao JSON (\textit{JavaScript Object Notation}). Tal ordenação e classificação é bem próxima de uma \textit{array} de objetos em linguagens orientadas a objeto, o que facilita a combinação das duas formas quando se exporta ou importa dados.

Outro ponto importante a ser citado se refere ao requisito geral 3. Usar uma API REST em Node.js abre caminho para fazer uma interface gráfica de usuário (GUI) para solicitar o arquivo de relatório. Neste momento ainda não foi decidido quais bibliotecas usar para criar a GUI.


\chapter{Desenvolvimento} \label{chdev}
\section{Ferramentas}

Para desenvolver código para o aplicativo, foi usado o Android Studio (Figura \ref{devfig02}), a IDE mais popular para esta tarefa, recomendada pelo próprio Google. Já para a API REST foi usado o editor de texto Atom (Figura \ref{devfig02}).

\begin{figure}[ht!]
	\centering
    \includegraphics[width=\textwidth]{Cap3/androidstudio}
    \caption{Interface do Android Studio}
    \label{devfig01}
\end{figure}

\begin{figure}[ht!]
	\centering
    \includegraphics[width=\textwidth]{Cap3/atom}
    \caption{Interface do Atom}
    \label{devfig02}
\end{figure}

\section{Protótipo 1 - BarcodeTest}

Este protótipo não chega a usar a API REST, que ainda não tinha sido feita. Ele é fortemente baseado no projeto que foi ensinado no curso \cite{kotlinandroid}, chamado HabitTrainer, mostrado na Figura \ref{devfig03}.

A Figura \ref{devfig03:figa} mostra um padrão de design Android chamado \textit{Recycler view}. Ele cria uma estrutura de cartões que otimiza o uso de memória por reciclar elementos que já estão na tela para produzir outros semelhantes. Tal estrutura foi usada no BarcodeTest para exibir os elementos cujo código de barras foi lido.

A figura \ref{devfig03:figb} mostra a tela de criação de novos hábitos no HabitTrainer. Este aplicativo usa a própria memória do celular com o padrão SQLite, que é portátil e fica armazenado no próprio celular. O BarcodeTest usa o mesmo modelo de banco de dados.

\begin{figure}[ht!]
	\centering
    \subfloat[Tela inicial][Tela inicial]{
    	\includegraphics[width=0.4\textwidth]{Cap3/habittrainer}
        \label{devfig03:figa}
    }
    \quad
    \subfloat[Adicionar novo hábito][Adicionar novo hábito]{
    	\includegraphics[width=0.4\textwidth]{Cap3/habittrainerbegin}
        \label{devfig03:figb}
    }
    \caption{Aplicativo HabitTrainer}
    \label{devfig03}
\end{figure}

\subsection{Código}

O BarcodeTest consiste de 3 \textit{Activities}, isto é, 3 ``telas'' diferentes, mostradas na Figura \ref{devfig04}. Cada uma delas será discutida a seguir.

\begin{figure}[ht!]
	\centering
    \subfloat[Tela inicial][Tela inicial]{
    	\includegraphics[width=0.3\textwidth]{Cap3/barcodetest}
        \label{devfig04:figa}
    }
    \quad
    \subfloat[Leitor de código de barras][Leitor de código de barras]{
    	\includegraphics[width=0.3\textwidth]{Cap3/barcodetestread}
        \label{devfig04:figb}
    }
    \qquad
    \subfloat[Inserindo os dados do material][Inserindo os dados do material]{
    	\includegraphics[width=0.3\textwidth]{Cap3/barcodetestinput}
        \label{devfig04:figc}
    }
    \quad
    \subfloat[Mensagem de erro][Mensagem de erro]{
    	\includegraphics[width=0.3\textwidth]{Cap3/barcodetesterror}
        \label{devfig04:figd}
    }
    \caption{Aplicativo BarcodeTest}
    \label{devfig04}
\end{figure}

\subsubsection*{\textit{MainActivity.kt}}

A \textit{MainActivity} (Figura \ref{devfig04:figa}) contém o padrão \textit{Recycler view} já citado, mostrando cartões para cada um dos códigos que foram lidos e salvos. Cada cartão representa um objeto da classe \textit{LoadMaterial.kt}, cujo código é descrito a seguir. Nesse código, pode-se ver uma das grandes vantagens de Kotlin com relação a Java, as \textit{data classes}. Elas provêm código pronto para \textit{getters}, \textit{setters}, \textit{toString} e outras funções que em Java seria necessário digitar uma a uma.

\begin{minted}[
frame=lines,
framesep=2mm,
baselinestretch=1.2,
fontsize=\footnotesize,
linenos,
label=LoadActivity.kt
]{kotlin}
package com.example.miguel.barcodetest

data class LoadMaterial(val code: String,
                        val description: String,
                        val section: String,
                        val mo: String)
\end{minted}

Nesse ponto, ainda não se usa a especificação de que as OMs e seções possuem um código numérico. Fica como o usuário digitou. A base de dados é especificada de acordo com o seguinte código SQL:

\begin{minted}[
fontsize=\footnotesize
]{sql}
CREATE TABLE material (id TEXT PRIMARY KEY, description TEXT, section TEXT, mo TEXT)
\end{minted}

Dessa forma, o id é o código de barras e, por ele ser único, ele é definido como chave primária da tabela. Para facilitar a leitura, armazenamento e comparação dos códigos, optou-se por colocar o código de barras como texto em vez de número.

Quando o usuário aperta o botão ``LER CÓDIGO'' no canto superior direito, a atividade é finalizada e inicia-se a atividade \textit{NewScanActivity}. Essa finalização é importante para atualizar a lista de cartões no caso de um novo código de barras ser lido e armazenado. Além disso, por causa da finalização que foi colocada tanto nessa quanto nas outras atividades, o botão de voltar quando apertado nessa tela sempre fecha o aplicativo.

\subsubsection*{\textit{NewScanActivity.kt}}

A \textit{NewScanActivity} (Figura \ref{devfig04:figb}) utiliza a biblioteca \textit{open-source} ZBarScanView, que possibilita a leitura dos códigos usando a caixa e a linha. É importante notar que para usar a câmera de um celular Android com um aplicativo, precisa-se pedir acesso explicitamente no arquivo \textit{AndroidManifest.xml} do projeto, da seguinte maneira:

\begin{center}
\mint{xml}|<uses-permission android:name="android.permission.CAMERA" />|
\end{center}

Ao ler um código de barras, esta atividade fecha e envia o número lido para a \textit{StoreScanActivity}. Se o botão de voltar for pressionado, esta atividade finaliza e o aplicativo abre uma nova \textit{MainActivity}.

\subsubsection*{StoreScanActivity}

A \textit{StoreScanActivity.kt} (Figura \ref{devfig04:figc}) recebe o código lido na \textit{NewScanActivity}. Tal código pode ser visto acima das caixas de texto. Ele é escrito para que o usuário confirme que o código foi lido corretamente, já que o leitor pode ter sido mal-posicionado em relação ao código ou algo do tipo. Com os dados inseridos, o botão ``SALVAR'' armazena o código na base de dados de acordo com o modelo que foi descrito.

Há duas funcionalidades importantes nesta atividade. A primeira é que, ao ler um código que já está salvo na base de dados, os campos que o usuário deve preencher já vêm preenchidos com o conteúdo armazenado relacionado a esse código, dando ao usuário a opção de atualizar os valores em vez de sobre-escrevê-los. Isto resolve o erro que poderia ser causado ao tentar salvar duas linhas na tabela com a mesma chave primária. A segunda é demonstrada na Figura \ref{devfig04:figd}, e consiste de exibir uma mensagem de erro caso algum dos campos esteja vazio, tornando assim os campos obrigatórios.

Ao salvar, esta atividade é finalizada e o aplicativo abre uma nova \textit{MainActivity}. Por outro lado, se o botão de voltar for pressionado, esta atividade finaliza e uma nova \textit{NewScanActivity} é aberta.

\subsection{Protótipo 2 - BarcodeMongo}

A grande diferença deste para o BarcodeTest é a inclusão da API REST. Aqui, é usada uma máquina virtual com Ubuntu 16.04 como servidor remoto. Usando a rede local, 

% This chapter explains how to set up SITL ArduPilot Simulator in a virtual machine environment on Windows following the instructions the Ardupilot Dev Team tutorial. The tutorial was tested on Windows 10 with Oracle VM VirtualBox 5.2.12 and Ubuntu 16.04.
% %http://ardupilot.org/dev/docs/setting-up-sitl-on-linux.html

% Software In The Loop simulator allows to run ArduPilot code without any hardware. So SITL simulation is ideal to test new featues and changes in the code during development.

% \section{Downloading the ArduPilot Code} \label{downloading_the_code}
% %http://ardupilot.org/dev/docs/setting-up-sitl-on-linux.html
% The ArduPilot project uses git for source code management and GitHub for source code hosting. To install git use the commands in the terminal:  %http://ardupilot.org/dev/docs/where-to-get-the-code.html#where-to-get-the-code

% \begin{verbatim}
% sudo apt-get update
% sudo apt-get install git
% \end{verbatim}

% Now you have to get a copy of the ardupilot git repository. Open the terminal and run:
% \begin{verbatim}
% git clone git://github.com/ArduPilot/ardupilot.git
% cd ardupilot
% git submodule update --init --recursive
% \end{verbatim}

% Install required packages using the script \textit{install-prereqs-ubuntu.sh}  located in \textit{ardupilot/Tools/scripts}. Go to the directory you cloned ardupilot into and use the following command line. It can take a while depending on your internet connexion. Don't forget to accept the changes when asked.
% \begin{verbatim}
% cd ardupilot/Tools/scripts/install-prereqs-ubuntu.sh
% \end{verbatim}

% Now you have to add the following lines to the end of your \textit{.bashrc} file in your home directory. Notice the . on the start of that filename. Also, this is a hidden file, so if you’re using a file manager, make sure to turn on “show hidden files”.

% \begin{verbatim} 
% export PATH=$PATH:$HOME/ardupilot/Tools/autotest
% export PATH=/usr/lib/ccache:$PATH
% \end{verbatim}

% Save the \textit{.bashrc} file and open the terminal. Then reload your PATH by using the “dot” command in a terminal:

% \begin{verbatim} 
% . ~/.bashrc
% \end{verbatim}
% If you don't change the  \textit{.bashrc} file you will not be able to use \textit{sim\_vehicle.py} as described in the next section.

% \section{How to start SITL simulator}
% In the terminal, go to the directory of the vehicle you want to make a simulation with. For example, for the multicopter code use the command line:
% \begin{verbatim}
% cd ardupilot/ArduCopter
% \end{verbatim}

% Then start the simulator using \textit{sim\_vehicle.py}. The first time you run it you should use the -w option to wipe the virtual EEPROM and load the right default parameters for your vehicle.
% \begin{verbatim}
% sim_vehicle.py -w
% \end{verbatim}

% After the default parameters are loaded you can start the simulator normally. First kill the sim\_vehicle.py you are running using Ctrl-C. Then:
% \begin{verbatim}
% sim_vehicle.py --console --map
% \end{verbatim}

% \section{FlightGear 3D View}  \label{sec_flightgear_view}

% It is possible to install FlightGear Flight Simulator to display a 3D simulation of the vehicle and its surroundings. To install FlightGear from terminal use next command line. It may take a few minutes to finish the download.

% \begin{verbatim}
% sudo apt-get install flightgear
% \end{verbatim}

% If you want to run the simulation including the FlightGear 3D view, you need to open a new terminal, go to the directory \textit{/ardupilot/Tools/autotest/} and open \textit{fg\_quad\_view.sh} (Copter). This will start FlightGear. The next steps show the procedure to run the simulation.

% \begin{enumerate}
% \item In a terminal use the command lines:
% \begin{verbatim} 
% cd ardupilot/Tools/autotest
% ./fg_quad_view.sh 
% \end{verbatim}

% \item In other terminal:
% \begin{verbatim} 
% cd ardupilot/Ardu
% sim_vehicle.py -j4 -L KSFO --map --console
% \end{verbatim}

% \end{enumerate}

% In the step 2, KSFO indicates the location where the simulation will take place. FlightGear will always start by loading scenery at KSFO (San Francisco International Airport) but will change to the selected scenery once SITL is started.
% If the vehicle appear to be hovering in space (no scenery) then you don't have the files for that location, but you can download it after. 

% \subsection{Adding a new location to FlightGear}
	
%     New locations are hard-coded into a file. This section shows how to add a new location. We will call the location as KSFO\_PAPI, because it will be next to PAPI lights in a lane of KSFO.
    
%     In the directory \textit{ardupilot/Tools/autotest} find the \textit{locations.txt} file. This file contains all the locations you can choose for simulation. Each line follows the sintaxe:

% \begin{verbatim}
% #NAME=latitude,longitude,absolute-altitude,heading
% \end{verbatim}  

% Then add to the file the next line. 
% \begin{verbatim}
% KSFO_PAPI = 37.6136, -122.357, 5.3, 297.9
% \end{verbatim}  
% When you run the simulation in KSFO\_PAPI location using the steps shown in section \ref{sec_flightgear_view}, FlightGear environment will be as in figure \ref{fig_papi_lights}.

% \begin{figure}[ht!] 
% \centering
% \includegraphics[width=1.0\textwidth]{Cap3/fig_papi_lights}
% \caption{Image from FlightGear simulator running in SITL. Observe four red lights (PAPI) in the left of the lane. }
% \label{fig_papi_lights}
% \end{figure}






\chapter{Próximos Passos} \label{chroadmap}
%\section{Conclusão}
After setting up the simulation environment and studying about MAVLink, the next steps is to check in the literature the main complains about MAVLink security in order to start to develop a more structured analysis. The analysis aims to exploit MAVLink vulnerabilities inside simulations in order to sustain conclusions.

During this process is expected to expand the MAVLink documentation this work want to present, including examples of normal use of the protocol in simulated environment. All the necessary steps to reproduce the simulations in other computers will be presented in order to help in related projects.


% REFERENCIAS BIBLIOGRAFICAS
\renewcommand\bibname{\itareferencesnamebabel} %renomear título do capítulo referências
\bibliographystyle{abnt-alf}
\bibliography{Referencias/referencias}

% Apendices
%\appendix
%\chapter{Topicos de Dilema Linear} %opcional
%\input{ApeA/apendiceA}

% Anexos
%\annex
%\chapter{Exemplo de um Primeiro Anexo} %opcional
%\input{AneA/anexoA}

% Glossario
%\itaglossary
%\printglossary

% Folha de Registro do Documento
% Valores dos campos do formulario
% \FRDitadata{ de Novembro de 2018}
% \FRDitadocnro{DCTA/ITA/DM-018/2015} %(o número de registro você solicita a biblioteca)
% \FRDitaorgaointerno{Instituto Tecnológico de Aeronáutica -- ITA}
%Exemplo no caso de pós-graduação: Instituto Tecnol{\'o}gico de Aeron{\'a}utica -- ITA
% \FRDitapalavrasautor{MAVLink; Drones; Software-In-The-Loop}
% \FRDitapalavrasresult{Cupim; Dilema; Construção}
%Exemplo no caso de graduação (TG):
% \FRDitapalavraapresentacao{Trabalho de Graduação, ITA, São José dos Campos, 2018. \NumPenultimaPagina\ páginas.}
%Exemplo no caso de pós-graduação (msc, dsc):
%\FRDitapalavraapresentacao{ITA, São José dos Campos. Curso de Engenharia da Computação. Área de Sistemas Aeroespaciais e Mecatrônica. Orientador: Prof.~Dr. Juliana de Melo Bezerra. Defesa em 29/11/2018. Publicada em 25/03/2015.}
% \FRDitaresumo{\input{Cap0/resumo}}
%  Primeiro Parametro: Nacional ou Internacional -- N/I
%  Segundo parametro: Ostensivo, Reservado, Confidencial ou Secreto -- O/R/C/S
% \FRDitaOpcoes{N}{O}
% Cria o formulario
% \itaFRD

\end{document}
% Fim do Documento. O massacre acabou!!! :-)
